{Current mass-urbanization trends create vast opportunities and new challenges to cities worldwide. Migration and immigration, climate change, technological disruptions, inequality, affordability, and most recently, global health concerns, are only some of the questions urban decision-makers are facing today. As these challenges grow, traditional urban processes are rendered insufficient: Slow planning processes lag behind rapidly expanding cities, static regulations struggle with volatile economies, and technological disruptions outpace legacy planning mechanisms.}

{In this dissertation I explore an emerging model, in which novel technologies and human-centric methods are coupled to create a \textit{new urban process}. This model marries a systematic, scalable, evidence-based and data-driven decision-making, with a comprehensive, long-term, and community driven participatory process. I explore this \textit{new urban process} through the design, development and deployment of CityScope: a human-centered, urban modeling, simulation and decision-making platform, on the intersection of urban technology and social discourse.}

{This dissertation examines past, current, and future CityScope instances through four perspectives: \textbf{Insight}: CityScope as an urban observatory, using advanced methods in urban and spatial analytics, focusing on high-resolution, geo-located urban-dynamics data.
\textbf{Transformation}: CityScope as an urban simulator for iterative design processes, and a dynamic system for decision-making.
\textbf{Prediction}: Using CityScope to support urban modelling in data-intensive and data-less environments, through methods of urban forecasting, modelling, and simulation.
\textbf{Consensus}: Enabling multi-stakeholder decision-making, consensus and policy processes. Using CityScope to facilitate collaborative and decentralized urban decision-making.} 

{Finally, I report on a series of lab experiments and worldwide deployments of the CityScope platform, and describe how it supported, enhanced, and even replaced traditional urban decision-making, affecting both the urban process as well as its outcomes.}