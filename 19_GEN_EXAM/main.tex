\section*{Primary Area:\\
A New Urban Process}
% 
\subsection*{Examiner} 
Kent Larson\\
Director, City Science Group, Media Lab
\\
Massachusetts Institute of Technology
\subsection*{Biography}
Kent Larson directs the City Science group at the MIT Media Lab. His research focuses on developing urban interventions that enable more entrepreneurial, livable, high-performance districts in cities. To that end, his projects include advanced simulation and augmented reality for urban design, transformable micro-housing for millennial, mobility-on-demand systems that create alternatives to private automobiles, and Urban Living Lab deployments in Hamburg, Andorra, Taipei, and Boston.\par
Larson and researchers from his group received the “10-Year Impact Award” from UbiComp 2014. This is a “test of time” award for work that, with the benefit of hindsight, has had the greatest impact over the previous decade. Larson practiced architecture for 15 years in New York City, with design work published in Architectural Record, Progressive Architecture, Global Architecture, The New York Times, A+U, and Architectural Digest. The New York Times Review of Books selected his book, Louis I. Kahn: Unbuilt Masterworks (2000) as one of that year’s ten best books in architecture.
% 
\subsection*{Description}

\textit{``For many cities (...) the pace of urbanization is overwhelming both national and local capacities to capitalize on the opportunities before them. The most common challenges include unplanned urban expansion, ineffectual governance and legal frameworks."} [World Economic Forum, 2016]
\par
The \textbf{main research area} explores the need for new urban processes amidst emerging challenges many cities are facing today. Readings in this section will explore areas in which traditional processes tend to fail or can no longer support current trends of urbanisation:
\begin{itemize}
    \item What renders traditional urban planning processes insufficient? 
    \item How do legacy regulatory and approval mechanisms inhibit urban development? 
    \item How data, analytics, evidences and predictions might make these processes more responsive?
\end{itemize}
In parallel, this section will explore the ways in which Smart Cities and urban-tech solutions address contemporary urban challenges:
\begin{itemize}
    \item What are the major Smart Cities solutions offered today? 
    \item Can these solutions fundamentally replace legacy urban systems or are only limited to system optimization?
    \item Can Smart Cities respond and curate to the needs of social groups and individuals in the city?  
\end{itemize}
Finally, readings in this section will explore emerging models which merge technology and social discourse in the form of a New Urban Process.
%%%%%%%%%%%%%%%%%%%%%%%%%%%%%%%%%%%%%%%%%%%%%%%%
\subsection*{Written Requirement}
The written requirement will consist of a 24-hour take-home examination to be administered and evaluated by Kent Larson.\par
%%%%%%%%%%%%%%%%%%%%%%%%%%%%%%%%%%%%%%%%%%%%%%%%
\textbf{Signature}\hspace{0.5cm} \makebox[2in]{\hrulefill}
%%%%%%%%%%%%%%%%%%%%%%%%%%%%%%%%%%%%%%%%%%%%%%%%
\newpage
\subsection*{Readings in the Primary Area}
%%%%%%%%%%%%%%%%%%%%%%%%%%%%%%%%%%%%%%%%%%%%%%%%
\subsubsection*{Challenges of Global Urbanization}
%%%%%%%%%%%%%%%%%%%%%%%%%%%%%%%%%%%%%%%%%%%%%%%%
\begin{bibunit}[unsrt]
\nocite{*}
\putbib[main/main_urbanization]
\end{bibunit}
%%%%%%%%%%%%%%%%%%%%%%%%%%%%%%%%%%%%%%%%%%%%%%%%
\subsubsection*{What are `smart cities'?}
\begin{bibunit}[unsrt]
\nocite{*}
\putbib[main/main_smartcities]
\end{bibunit}
%%%%%%%%%%%%%%%%%%%%%%%%%%%%%%%%%%%%%%%%%%%%%%%%
\subsubsection*{Measuring Cities}
\begin{bibunit}[unsrt]
\nocite{*}
\putbib[main/main_indicators]
\end{bibunit}
%%%%%%%%%%%%%%%%%%%%%%%%%%%%%%%%%%%%%%%%%%%%%%%%
\subsubsection*{Towards A New Urban Process}
\begin{bibunit}[unsrt]
\nocite{*}
\putbib[main/main_urbanscience]
\end{bibunit}