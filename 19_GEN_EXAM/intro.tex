\section*{Introduction}
Over 50\% of the world population currently lives in urban areas, projected to grow to 68\% by 2050 \cite{united2018world}. In comparison, urban growth rates of the past three decades (\%39 - \%50) are nearly equal to the entire trend of urbanisation between 1650 - 1900, \%5.9 - \%18 \cite{grauman1976orders, chen2014global}. This rapid pace of urbanization creates new opportunities as well as new challenges: Urban migration and immigration, climate change, technological disruptions, inequality and affordability are only some of the issues facing urban decision makers today \cite{parnell2016defining, reckien2017climate}. As these challenges grow, traditional urban processes are rendered insufficient: Slow moving planning procedures lag behind rapidly expending cities; static regulatory systems often struggle with changing economics and technological disruptions; Decision-makers suffer from scarcity of data and evidence-based processes \cite{world2016inspiring}.\par
In recent years, tech entrepreneurs and Smart Cities advocates have taken advantage of the gaps created by traditional urban processes. The Smart Cities movement introduced solutions that were scalable, fast to deploy, data driven and evidence based \cite{soderstrom2014smart}. Yet instead of dealing with urban issues in a comprehensive way, many of these new technologies tend to offer highly-specific optimizations to legacy systems. The focus on optimization often overlooks the unique and ever-changing needs of groups and individuals in the city, thus offering only short term solutions \cite{gaffney2018smarter}.\par
In this respect, both traditional urban processes as well as contemporary Smart Cities solutions cannot efficiently respond to the emerging challenges of 21st century cities \cite{soderstrom2014smart}. In this doctoral examination, I explore an emerging model in which novel technologies and traditional processes are coupled to create a \textit{new urban process}. From the world of Smart Cities and urban-tech, this model inherits the systematic, scalable, evidence-based and data-driven approach. From traditional urban processes, this model adopts the comprehensive, long-term, and community driven attitude. This new urban process was manifested in the New Urban Agenda proposed by the UN-Habitat:  
{\textit{``The New Urban Agenda presents a paradigm shift based on the science of cities; it lays out standards and principles for the planning, construction, development, management, and improvement of urban areas (...) We will foster the creation (...) of open, user-friendly and participatory data platforms using technological and social tools available to transfer and share knowledge (...) to enhance effective urban planning and management."}} \cite{habitat2016new}
\par
My research examines this new model through the design, development and deployment of CityScope. CityScope is a human-centered, urban modeling, simulation and decision-making platform that merges urban technology and social discourse. This doctoral examination will expend on a \textit{CityScope Process} through four major themes:
\\
\textbf{Insight}: urban observations through data gathering, analysis and communication 
\\
\textbf{Predictions}: modeling urban insights into effective predictions
\\
\textbf{Transformations}: urban-design and spatial 'what-if' scenarios through iterative cycles 
\\
\textbf{Consensus}: constructing multi-stakeholder decision-making process, consensus and agreement 
\\
The purpose of this examination is to help me construct a wide knowledge-base around the past, current and future implementations of CityScope. It will help me classify CityScope's contributions within each of the four themes, and will provide the scientific basis for further development of the platform. 
The literature review is composed of three sections: The \textbf{Main} section explores recent challenges cities and decision-makers are facing. It further expands on the Smart Cities movement, its contributions and perils, and concludes with readings exploring the emergence of a New Urban Process. \\
The \textbf{Technical} section discusses \textit{Insight} and \textit{Prediction} as deferential aspects of future urban processes. It expends on current advanced methods in urban and spatial analysis, focusing on high-resolution, geolocated data. It concludes with an overview on recent methods in foresighting urban dynamics.   
Finally, the \textbf{Contextual} section explores the development of tools, methods and processes for iterative and exploratory urban \textit{Transformation}. It expend on socio-technical systems that can facilitates \textit{Consensus} and concludes with recent developments in collaborative and distributed decision making systems for urban planning. 
% 
\subsection*{Introduction References}
\bibliographystyle{plain}
\bibliography{intro/intro_bib}
