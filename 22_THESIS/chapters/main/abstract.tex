{
Current mass-urbanization trends create vast opportunities alongside new challenges to cities worldwide. Immigration, climate change, technological disruptions, inequality, and health concerns, are only some of the questions urban decision-makers are facing today. As these challenges grow, traditional urban processes are rendered insufficient, as they trail behind rapidly expanding cities and technological disruptions.
\newline
In this dissertation I investigate a new urban process, which couples data-driven and evidence-based decision-making, with human-centric and participatory planning. I explore this new urban process through the design, development and deployment of \textbf{CityScope}: an urban modeling, simulation, and decision-making platform. From collaborative allocation of refugee-housing in Germany, through crowd-sourced mapping of public safety in Guadalajara, to mass-transit co-creation in Boston, CityScope helps to build agency amongst the `have-nots', who traditionally were denied from the urban process.
\newline
I report on a series of lab experiments and real-world deployments of CityScope through four themes: \textbf{Insight}: CityScope as an urban observatory, using real-time spatial data and urban dynamics analytics; \textbf{Transformation}: CityScope as an iterative, collaborative, and real-time Urban Human Computer Interaction system; \textbf{Prediction}: CityScope for urban forecasting and simulation of implicit aspects in the built environment; and \textbf{Consensus}: CityScope for collaborative decision-making with diverse stakeholders and communities.
\newline
Finally, I describe how CityScope supported, enhanced, and occasionally replaced traditional urban decision-making, affecting both the urban process as well as its outcomes.
}

