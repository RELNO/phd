\chapter*{Conclusions}\label{ch:conclusions}
\addcontentsline{toc}{chapter}{Conclusions}

{
    \textbf{By developing, testing, and globally deploying CityScope, this thesis demonstrated how Urban Human-Computer Interaction platforms can support a new urban process. This process brings together modern technologies alongside social, cultural, and political discourse, to inform an evidence-based yet collaborative urban decision-making.} In that respect, CityScope aims to tackle some of the most fundamental limitations of current urban processes and planning tools: It addresses the growing divide between highly-optimized yet tecnho-centric design processes, and community-driven yet slow and ineffective city-planning. It aims to bring together the best of both worlds, so that a community-focused, yet evidence-based planning process could emerge. Motivated by efforts to tackle mass-urbanization, climate change, energy shortages, social, economic, and political turmoils, CityScope is structured around a multi-phase and open-source approach to collaborative planning. These phases and their application as part of the CityScope process are summarized next.
    \newline
    Chapter~\eqref{ch:insight} introduced the potential of actionable insights to inform urban decision-making, ranging from mobility, energy, economic, social, and cultural observations. It focused on insights that are the product of many different observations from variety of data sources, so that anomalies, edge-cases, and relevant questions would surface. It then demonstrated ways to collect, process, represent, and communicate these observations through the CityScope platform, as a mean to bootstrap an urban-transformation process.
    \newline
    Chapter~\eqref{ch:transformation} introduced the core design principles of the CityScope platform, and the ways it supports real-world urban transformation processes. It reviewed the methodologies, systems, and interfaces composing the CityScope ecosystem, and emphasized the open-ended design of the platform, and the interoperability of urban data and spatial analysis using the CityScope Schema. It then demonstrated the way CityScope supports urban transformation through the analysis of different metrics, from formal, spatial, and environmental to economic, cultural, and social. It concluded with several case-studies of real-world urban transformation processes, and the ways CityScope supported them.
    \newline
    Chapter~\eqref{ch:prediction} considered more abstract ways by which CityScope can support decision-making, through the modeling and prediction of implicit aspects of the built environment. It reviewed the range of urban modeling and prediction methods used in current planning and development processes, and suggested a plausible future for urban models which stretches beyond tactile, spatial, and environmental topics. It then demonstrated how more `subtle' observations, such as human dynamics, social gathering, and visual perception, might evolve into implicit urban models, and how these can support a more inclusive planning process.
    \newline
    Finally, Chapter~\eqref{ch:consensus} poised CityScope in the center of consensus-driven urban decision-making. It emphasized the need to coverage the advantages of urban technologies with a public-facing social discourse. It focused on ways to create platforms, methods, models, and data that are inherently open for review, evaluation, and criticism, as a mean to establish a technological consensus, true partnership, and citizen control.
    It concluded with key case-studies demonstrating CityScope as a tools for public-participation, co-creation, learning, and collaborative urban decision-making.


    \section{From Smart Cities to Street Knowledge}
     {
      \textit{How should a New Urban Process look like?} This closing section discusses this thesis contributions with a broader perspective and potential future follow-ups and applications. It aims to draw a plausible future, in which the fundamental ideas behind CityScope find their way into the mainstream of city-planning and urban development. Some of the visions mentioned here are under active development, or even already in use, where others are still years ahead; Nevertheless, the future of urban processes is a complex and interdependent effort, and progress can be achieved by a broad range of approaches.
      \newline
      To exemplify this `New Urban Process', this section explores a scenario in which a fictional city is considering its expansion amidst growing waves of newcomers. The city considers alternatives, such as rapid development of its outskirts, densification of lightly- and underbuilt areas in its center, repurposing of vacant and underused land, or conversely - adopting a passive approach, and adhering to market forces with minimal intervention. In the recent decade, this city has adopted a set of `New Urban Process' strategies to address such challenges, and is now aiming to implement them in a real-world use-case.


      \subsection{City State}
      {
          The first step is a rapid and comprehensive understanding of the current state of affairs. Unlike past processes, in which lengthy and costly consultation was needed, the city is already aware of its current state through a set of real-time, up-to-date observations. These include current housing stocks, real-time population, current and past traffic, energy demands, water, waste management, as well as other environmental factors. The city enjoys a real-time yet anonymized census of its current population and visitation frequency, and is able to assess the current activity levels of both its residents as well as of commuters and visitors.
          \newline
          In addition, the city holds a live record of all of its physical assets, such as buildings, infrastructure, or landscape, and is able to provide a complete picture of its current physical environment at any given or past time. A real-time Digital Twin is continuously being updated to include the current state of the city, as well as new plans, developments, policies, and urban processes as they progress. Upcoming changes would be reflected on a shared, open-source ledger, which is securely immutable and available to all users. It will follow open standards which are adaptable by other cities and regions, so that each city could learn and evaluate from the best practices of other cities. As discussed later, changes to these ledgers would only occur by a consensus-driven process, open for public review and scrutiny.
          \newline
          Urban data are openly collected and made available to the public. The city itself utilizes aggregated and anonymized data collection from various sources, both static and dynamic, which are fully consented by citizens, and are open to on-going public review and evaluation. Private data, such as individual human movement or specific social interaction cannot be obtained by the city, and only simulated data would become available. Unlike the past, in which the city had to rely on 3\textsuperscript{rd} party data provides, the city now has developed the ability to securely collect and analyze data from various sources, while adhering to the highest standards of privacy and security. The city is also participating in a coalition of several other cities, to share and review data collected in their municipalities, so that events and trends can be evaluated beyond the regional bounds.
      }

      \subsection{Iteration, Generation, and Adaptation}
      {
          As the challenge of newcomers becomes more pressing, the city's real-time data-analytics system highlights the tradeoffs of this challenge using various indicators. for example, it points to the increase in rent in some areas, while also highlighting the growth in retail transactions in other parts of the city. At the same time, an iterative pre-planning process is self initiated; This process runs in parallel to the current stream of urban data, and is responsible to create constant predictions of possible futures scenarios. This system is designed to be iterative, and is able to adapt to changes in the state of the city as they occur. At this point, decision-makers and the general public can start query the system for possible urban transformations, predictions, and assessments of scenarios, as a baseline for future public discussion. Unlike pre-planning processes of the past, the city does not have to intentionally initiate a planning study, a survey, or commission an exploratory process, as these are self-initiated as soon as a new urban challenge is identified.
          \newline
          For each of the plausible planning scenarios examined by the system, a comprehensive set of analyses, metrics, KPIs, and indicators is provided. Each of these is open-sourced and well documented, to enable the city and residents a better judgment of the quality of the proposed plan, as well as the mechanisms by which it was assessed. These metrics are also easily extendable and customizable, and can be used to create new indicators and KPIs, so that other ways to evaluate the city's future can be explored. As this step concludes, the city is setting itself towards a public participation process, where pre-planning ideas would be considered by a variety of stakeholders.
      }

      \subsection{Making Decision-Making}
      {
          The participatory process utilizes the same underlining system described before, already including data, models, scenarios, and assessments. This phase aims to marry the pre-assessments made autonomously using the city's planning mechanisms, with the public's input. In the center of this deliberation, are physical and regulatory changes that can affect the form of parts of the city, as well as land-use properties and functional aspects of the buildings in this area. In the longer term, these changes will also have resonating effects on local economy, real-estate values, and social and cultural aspects.
          \newline
          At the inception stage, an open-source collaboration platform would become available to the general public. Easy to use and accessible by design, this tool would allow citizens to observe a set of prevailing proposals in their urban context, as well as discuss, comment, and vote for their preferred alternatives. This tool will include  a subset of the urban KPIs and metrics, which were appropriated to the evaluation of this specific proposal. The application interface would be built for any digital platform, and will provide the same user access, level of detail and interactivity to all users.
          \newline
          In addition, `citizen design' sessions would be offered using the same underlining tools, allowing participants to not only observe predetermined proposals, but collaboratively donate their own \textit{street knowledge}. Using a distributed ledger, all design alternatives, both conceived by professionals as well as those co-created by citizens, would be collected and analyzed through the same set of KPIs. Clustering techniques would merge similar proposals by their performance adjacency and spatial attributes, to reflect areas of interest and design `hot-spots'. Other mechanisms will expose edge-cases revealed by both the public and professionals, to be further assessed by the city.
          \newline
          Lastly, in parallel to a participation process for the general public, citizens and groups directly affected by the proposal (due to geographic proximity or other affiliations), would be selectively approached to share their input. On set dates, dedicated venues would host consensus building sessions, with experts, focus groups and city-supported moderators. Results from these sessions would be highlighted in the aggregated participation-data layer in the city's ledger.
      }

      \subsection{Participatory Review Session}
      {
          Upon completion of both the general and focused participation phases, members of the city-planning authority, stakeholders, and the public representatives, would attend a decision-making session. In this milestone event, an instance of the Urban HCI platform will portray the prevailing subset of selected planning alternatives from previously aggregated proposals. During the course of this session, representatives and delegates would use the platform to highlight concerns, observe notable alternatives, and discuss tradeoffs. Their interactions would be streamed and recorded into the city's distributed ledger, for later evaluation, validation and feedback.
      }

      \subsection{Follow-up and Validation}
      {
          Lastly, a voting session will conclude the urban process, and the selected proposal would be affixed to the Urban Version Control ledger. After concluding the participation phase, both approved and rejected proposals will remain in the system, allowing citizens to follow implantation, compare deliverables, and validate initial predictions. Simultaneously, a follow up process would be initiated to interested parties, including progress reports, implementation efforts, and challenges as they arise.
          \newline
          In the future, the increments between identifying urban challenges and initiating a participatory planning process would reduce to minimal. Urban decision-making would not happen only through stepped milestones and major master-plans, but could gradually phase out traditional planning processes, by shifting from Phased Implementation to a continuous decision-making. Clearly, embracing such a complex \textit{urban operating system} might not be straightforward, but this effort could simplify and improve costly, lengthy and overly-bureaucratic processes, while addressing the needs, concerns, and challenges of urban dwellers.
      }
     }

    \section{Future Outlook}
     {
      Some of the ideas detailed in the use-case above were implemented in past instances of CityScope; Others are beyond the scope of this work and could be part of future exploration. The rest of this section details some potential follow-ups and applications of CityScope in several areas:

      \subsection{Iterative Design}
      {
          \textbf{Decision Support System:} CityScope contributes to Decision Support Systems by reducing the burden and time between design iterations and simulation results. Future version could include a more robust interface between urban data-exploration, iterative search for solution, and simulation results, so that design questions could be answered in a more intuitive way. A streamlined method to plug-in arbitrary data sources and new analysis modules would reduce the set up procedure of new CityScope instances as well as allow for more flexibility in the design and evaluation process.
          \newline
          \textbf{Optimization Models and `AI' Support:} With the goal of providing near real-time simulation results, CityScope interface could be extended to include different optimization methods, which would be used to generate a set of simulation results ahead of user interaction. This could be achieved by using data-driven models which would be trained on many random design alternatives and their impacts on various analyses, such as noise, water, wind, or traffic \cite{zhang2017citymatrix, dnsl19}. This approach could support both real-time `ballpark' estimations, while allowing more accurate models to finalize high-end analysis in the backend.
          \newline
          \textbf{Generative Design:} CityScope interface could be extended to include generative algorithms, which will be used to create a preset of planning alternatives, and optimize them to a dynamic array of KPIs and urban metrics \cite{DelvebyS33:online}. In keeping with CityScope core principles of user-interaction and participation, generative design should be integrated so that both users and machines could interact with the same set of planning alternatives as equal counterparts in a shared design space.
      }

      \subsection{Interaction and Interface}
      {
          As discussed in Chapter \eqref{ch:transformation}, CityScope interfaces are designed to be flexible and extensible beyond the traditional TUI. When technologies such as AR, VR, or MR will mature to become seamless, intuitive, and non-obstructing to the user, urban simulation could greatly benefit from the ability to interact with physical-virtual environments \eqref{sec:cityscope_ar}.
      }

      \subsection{Transformation and Interventions}
      {
          Current CityScope design promotes the exploration of mostly spatial transformations, such as new urban, infrastructure, and transportation developments. Future CityScope could also include a broader spectrum of interventions, such as new energy solutions (solar, fusion, or other renewable sources, as well as their impact on urbanization patterns and urban mobility); alternative delivery and logistic systems (such as light-weight, autonomous, and shared last-mile mobility solutions); new economic models (such as fractional real-estate ownership and decentralized micro-economies at the neighborhood scale); as well as environmental and climatical changes. For example, the development of a new neighborhood, which might only be completed within two decades, could use CityScope to assess the local tradeoffs of climate change, greenhouse gas emissions, or sea-level rise, and dynamically adapt to these new environmental factors.
      }

      \subsection{Governance}
      {
          \textbf{New Metrics:} As discussed in Chapter \eqref{ch:prediction}, more implicit and subtle metrics could greatly enhance the evaluation of future urban transformation. These metrics could affect policy and regulatory considerations, and could impact decisions stretching beyond the confines of any single intervention. Future CityScope platforms could be used to analyze these metrics, and to provide a more holistic view of human dynamics and spatial-behavioral patterns.
          \newline
          \textbf{Simulating Public Policy:} CityScope modules could be extended to explore more abstract transformations, such as new rules, regulations, and public policy. These modules would account for the many underlines impacts of new policies, and could be designed to answer complex questions such as `what is the impact of a new policy?'
          \newline
          \textbf{Dynamic Zoning:} Beyond the evaluation of urban and regional transformation, CityScope could become a tool for controlling `Dynamics Zoning' (see \eqref{sec:cityscope_playground}) \cite{DynamicZ48:online}. Dynamics Zoning refers to agile urban regulatory mechanisms that are self-adjusting based on community-driven, agreed-upon indicators and decision triggers. This is in contrast to traditional zoning, where the planning system is rigid, and the process of amending is slow, bureaucratic, and heavily dictated by the city. With Dynamics Zoning, CityScope could be a central hub for testing, validation, voting, and collecting feedback of planning adjustments. 
      }


      \subsection{Informality and Underprivileged Communities}
      {
          Given their physical-technological nature, early CityScope instances were mostly designed as one-offs, costly, and non-scalable installations. These were hard to adapt to the needs of new urban communities, and were often limited to a single city; Naturally, these limitations made CityScope more applicable to affluent `first world' cities. In recent years, the majority of CityScope software became open-source and publicly available, allowing for less privileged cities and academic institutions to develope their own custom CityScope software \footnote{The majority of CityScope projects and tools are developed under gpl-3.0 license at https://github.com/CityScope.}. Nevertheless, the specialized hardware needed to run CityScope was still expensive and required expertise in various fields. CityScopeJS (see Section \eqref{sec:cityscope_architecture}) is a browser-based, open-source CityScope software, which was designed to be used on any device, without the need for specialized hardware or software. After its debut in 2019, CityScopeJS has been used by a wide range of academic institutions, government agencies, and non-profit organizations around the world to create and analyze urban planning scenarios.
          \newline
          \textbf{Kit-of-Parts:} As CityScope moves from a series of one-off solutions to a more general-purpose platform, it should address the needs and limitations of less privileged communities around the world, and provide a more comprehensive set of tools for future urban-planning. Future development could offer a kit-of-parts to CityScope processes, where each of the different components of the process are available and accessible to any community. For example, methods of reaching local-leadership and relevant stakeholders (such as in the BRT and FindingPlaces projects, see Sections \eqref{sec:brt}, \eqref{sec:findingplaces}) could become more generic and offer structured mitigation and outreach processes; Methods to collect and analyze local data as part of `Insights', could be simplify and standardized, so that even local communities could use them to inform their own planning decisions.
      }
     }


    \bigskip
    \begin{center}
        $\sim$
    \end{center}
    \bigskip
    {
        Finally, it is my belief that an evidence-based and participatory planning has the potential to change the way we live, work, and experience our built environment. I hope that \textbf{CityScope}, which began as a playful experiment and grew to become a world-wide methodology for urban decision-making, will continue to be a useful tool for the betterment of cities, and the creation of a more participatory, evidence-based, and sustainable urban future.
    }

}


